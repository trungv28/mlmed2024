\documentclass{article}
\usepackage{graphicx}
% \usepackage{float}
\title{Report on ECG hearbeat classification}
\author{Tran The Trung}
\begin{document}
\maketitle

\section{Introduction}

The heart is a organ that primarily in charge of pumping blood throughout the body, 
providing oxygen and nutrients to various tissues and organs. The heart is basically
recognized as the living source for every live beings in the planet and human is obviously
not an exception.\\

\noindent The heartbeat, on the other hand, refers to one complete pulsation of a heart
It is the regular movement of your heart as it sends blood around the body and it happens
from birth till death .\\

\noindent And with the ever-increasing pace of AI or more specifically Machine Learning, people
have been able to enhance diagnosis, improve treatment and streamline healthcare.
And if we take a closer look into heartbeat classification, ML algorithms have been
employed to classify heartbeats into different categories, making foundation for
solution of various problems.\\
 
\section{Background}
The data is about ECG heartbeat which composes of two hearbeat signals derived
from two famous datasets in heartbeat classification, the MIT-BIH Arrhythmia
Dataset and the PTB Diagnostic ECG Database. The signals correspond to
electrocardiogram (ECG) shapes of heartbeats for the normal case and the cases
affected by different arrhythmias and myocardial infarction. These signals
are preprocessed and segmented, with each segment corresponding to a heartbeat.\\

\noindent In this day, Convolutional Neural Network is one of the most popular
Deep Learning model which is primarily used in Computer Vision domain to extract
features from visual data such as images or videos. However, many reasearchs have shown
that CNN also work well on 1D data or in this case, the classification of
ECG heartbeat.

\section{Method}

\begin{table}[h]
    \centering
    \begin{tabular}{|l|l|l|l|l|}
    \hline
    Type        & \# Channels & Kernel Size & Stride & Padding \\ \hline
    Convolution 1d & 32          & 3           & 1      & 1       \\ \hline
    Max Pool    &             & 2           & 2      & 0       \\ \hline
    Convolution 1d & 64          & 3           & 1      & 1       \\ \hline
    Max Pool    &             & 2           & 2      & 0       \\ \hline
    3 FC layers   &             &             &        &         \\ \hline
    \end{tabular}
    \end{table}

I use a convolutional neural network with two convolution layers followed with
a max pool layer and a ReLU activation function. After extracting features,
I use fully connected layers to classify the signal.\\

To explain the choice of architecture, using convolution helps scan the input
signal with small, trainable filter and this acts as feature detectors. This type
of layers play the same purpose as the original 2d convolution where it identifies
basic patterns. The pooling layers are used to reduce the size of the signal or more
specifically its channels. This is crucial to reduce the amount of parameters and
computation while increase the robustness of the model. And ReLU activation function
is there to add some non-linearity to the model which is the main purpose of a
Deep Learning model.At the end of the network, fully connected layers perform high-level
reasoning by combine these features and make the final decision about the class of the objec

\section{Evaluation}

\subsection{Dataset}

\noindent In this practical section, I will only focus on the first data which is the 
MITBIH data. This data contains the total of 109446 samples which are then divided
into 87554 and 21892 sample for training and test set, respectively. There are in
total 5 classes: N, S, V, F, and Q. All the samples in the data have already been
cropped, downsampled and padded with zeros if necessary so that each of them has
a fixed dimension of 188, with the latest column being the class.

First, we need to visualize some aspects regarding the data.

\begin{figure}[h]
    \centering
    \includegraphics[width=0.7\textwidth]{data_dist.png}
    \caption{Data distribution before over-sampling}
\end{figure}

The figure 1 below show that the data is highly imbalanced so we need to 
make steps to balance out the data. I have done this by over-sampling the data
using SMOTE technique which increase the sample from all the minor class to be
equal to the majority one.

\begin{figure}[h]
    \centering
    \includegraphics[width=0.7\textwidth]{data_dist2.png}
    \caption{Data distribution after over-sampling}
\end{figure}

\subsection{Result}
The model achieve a 98 \% accuracy which is very high for 5-class classification
problem. This show how powerful CNN model is to 1D data since it enables
the ability of learning neighbor relationship between features in the data.

\begin{figure}[h]
    \centering
    \includegraphics[width=0.7\textwidth]{conf_mat.png}
    \caption{Confusion Matrix}
\end{figure}
From the confusion matrix, it is seen that the model perform well on every classes except
for the the class 3. This shows that our model performs quite good overall and the 98 \%
is not a result of a over-sampling procedure

\section{Discussion}
The extremely high accuracy might not be that great since there's might be some
problems in training like overfit, imbalanced data or it might simply due to errors
in code and I need to conduct further investigation on this.

\end{document}